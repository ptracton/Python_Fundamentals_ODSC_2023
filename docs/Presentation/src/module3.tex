\section{Python Fundamentals Module 3}

\subsection{Lecture Summary}
\begin{Slide}{Lecture Summary}
  \begin{itemize}
  \item import
  \item numpy
  \item Matplotlib
  \item pandas
  \end{itemize}
\end{Slide}

%%%%%%%%%%%%%%%%%%%%%%%%%%%%%%%%%%%%%%%%%%%%%%%%%%%%%%%%%%%%%%%%%%%%%%%%%%%%%%%% 
% 
% 
% 
%%%%%%%%%%%%%%%%%%%%%%%%%%%%%%%%%%%%%%%%%%%%%%%%%%%%%%%%%%%%%%%%%%%%%%%%%%%%%%%% 
\begin{Slide}{Import}
  \begin{itemize}
  \item The \underline{\href{https://docs.python.org/3/reference/import.html}{Python Import System}}
  \item This is how we bring in code from one module to another
  \item Gain access to built in python libraries
  \item Gain access to downloaded 3rd party libraries
  \item Create your own to organize your code
  \end{itemize}
\end{Slide}

\subsection{Numpy}
%%%%%%%%%%%%%%%%%%%%%%%%%%%%%%%%%%%%%%%%%%%%%%%%%%%%%%%%%%%%%%%%%%%%%%%%%%%%%%%% 
% 
% 
% 
%%%%%%%%%%%%%%%%%%%%%%%%%%%%%%%%%%%%%%%%%%%%%%%%%%%%%%%%%%%%%%%%%%%%%%%%%%%%%%%% 
\begin{Slide}{Numpy}
  \begin{itemize}
  \item  Numerical Processing library for python
  \item Lots of tools are built on top of this
  \item It is a large library with lots of \underline{\href{ttps://numpy.org/learn/}{guides}} and features
  \item ndarrays can be multi-dimensional
  \end{itemize}
\end{Slide}

%%%%%%%%%%%%%%%%%%%%%%%%%%%%%%%%%%%%%%%%%%%%%%%%%%%%%%%%%%%%%%%%%%%%%%%%%%%%%%%% 
% 
% 
% 
%%%%%%%%%%%%%%%%%%%%%%%%%%%%%%%%%%%%%%%%%%%%%%%%%%%%%%%%%%%%%%%%%%%%%%%%%%%%%%%% 
\begin{Slide}{ndarray}
  \begin{columns}
    \begin{column}{.48\textwidth}
      
      \begin{itemize}

      \item This is the \underline{\href{https://numpy.org/doc/stable/reference/generated/numpy.ndarray.html}{key datatype}} of numpy.
      \item Similar to a list but expects all data to be numeric and the same type.
      \item You can create an ndarray from lists 
      \item Lots of features to \underline{\href{https://numpy.org/devdocs/user/basics.creation.html}{ndarray}}
        
      \end{itemize}
      
      
    \end{column}

    \hfill

    \begin{column}{.48\textwidth}
      Code Example:
      
    \inputminted[firstline=3,
      lastline=19,
      breaklines,
      fontsize=\tiny,
      bgcolor=Background,
      linenos]{python}{../src/ndexample.py}
      
      
    \end{column}
  \end{columns}

\end{Slide}


\subsection{Matplotlib}
%%%%%%%%%%%%%%%%%%%%%%%%%%%%%%%%%%%%%%%%%%%%%%%%%%%%%%%%%%%%%%%%%%%%%%%%%%%%%%%% 
% 
% 
% 
%%%%%%%%%%%%%%%%%%%%%%%%%%%%%%%%%%%%%%%%%%%%%%%%%%%%%%%%%%%%%%%%%%%%%%%%%%%%%%%% 
\begin{Slide}{Matplotlib}
  \begin{itemize}

  \item One of many plotting libraries for Python
      \begin{itemize}        
      \item \underline{\href{https://matplotlib.org/}{Matplotlib}}
      \item \underline{\href{https://bokeh.org/}{Bokeh}} 
      \item \underline{\href{https://seaborn.pydata.org/}{Seaborn}}
      \item \underline{\href{https://plotly.com/}{Plotly}} 
  \end{itemize}
  
\item Comes with a lot of documentation
  \begin{itemize}
  \item \underline{\href{https://matplotlib.org/stable/plot_types/index.html}{Examples}} 
  \item \underline{\href{https://matplotlib.org/cheatsheets/}{Cheat Sheet}} 
  \item \underline{\href{https://matplotlib.org/stable/users/getting_started/}{Getting Started}} 
  \end{itemize}

    
  \end{itemize}
\end{Slide}

%%%%%%%%%%%%%%%%%%%%%%%%%%%%%%%%%%%%%%%%%%%%%%%%%%%%%%%%%%%%%%%%%%%%%%%%%%%%%%%% 
% 
% 
% 
%%%%%%%%%%%%%%%%%%%%%%%%%%%%%%%%%%%%%%%%%%%%%%%%%%%%%%%%%%%%%%%%%%%%%%%%%%%%%%%% 
\begin{Slide}{Basic Plotting}
  \begin{columns}
    \begin{column}{.48\textwidth}
      
      \begin{itemize}

      \item  \underline{\href{https://matplotlib.org/stable/api/\_as\_gen/matplotlib.pyplot.plot.html}{Plotting}}
      \item plot and pair of x and y values.
      \item add to the \underline{\href{https://matplotlib.org/stable/api/_as_gen/matplotlib.pyplot.legend.html}{legend}} in the order added to the plot
      \item Can set \underline{\href{https://matplotlib.org/stable/api/_as_gen/matplotlib.pyplot.xlabel.html}{x}} and \underline{\href{https://matplotlib.org/stable/api/_as_gen/matplotlib.pyplot.ylabel.html}{y}} axis labels
      \item Can set a \underline{\href{https://matplotlib.org/stable/api/_as_gen/matplotlib.pyplot.title.html}{title}}
      \item \underline{\href{https://matplotlib.org/stable/api/_as_gen/matplotlib.pyplot.grid.html}{Grid}} can be enabled or disabled
      \item Need to show the plot to get it to display

      \end{itemize}
      
      
    \end{column}

    \hfill

    \begin{column}{.48\textwidth}
      Code Example:
      
    \inputminted[firstline=3,
      lastline=23,
      breaklines,
      fontsize=\tiny,
      bgcolor=Background,
      linenos]{python}{../src/plot1.py}
      
      
    \end{column}
  \end{columns}

\end{Slide}

%%%%%%%%%%%%%%%%%%%%%%%%%%%%%%%%%%%%%%%%%%%%%%%%%%%%%%%%%%%%%%%%%%%%%%%%%%%%%%%% 
% 
% 
% 
%%%%%%%%%%%%%%%%%%%%%%%%%%%%%%%%%%%%%%%%%%%%%%%%%%%%%%%%%%%%%%%%%%%%%%%%%%%%%%%% 
\begin{Slide}{SubPlots}
  \begin{columns}
    \begin{column}{.48\textwidth}
      
      \begin{itemize}

      \item You can also visualize \underline{\href{https://matplotlib.org/stable/gallery/subplots_axes_and_figures/subplots_demo.html}{multiple plots}} in the same image.
      \end{itemize}
      
      
    \end{column}

    \hfill

    \begin{column}{.48\textwidth}
      Code Example:
      
    \inputminted[firstline=3,
      lastline=23,
      breaklines,
      fontsize=\tiny,
      bgcolor=Background,
      linenos]{python}{../src/plot2.py}
      
      
    \end{column}
  \end{columns}

\end{Slide}

\subsection{Pandas}
%%%%%%%%%%%%%%%%%%%%%%%%%%%%%%%%%%%%%%%%%%%%%%%%%%%%%%%%%%%%%%%%%%%%%%%%%%%%%%%% 
% 
% 
% 
%%%%%%%%%%%%%%%%%%%%%%%%%%%%%%%%%%%%%%%%%%%%%%%%%%%%%%%%%%%%%%%%%%%%%%%%%%%%%%%% 
\begin{Slide}{Pandas}
  \begin{itemize}
    \item \underline{\href{https://pandas.pydata.org/}{Pandas Library}}
    \item Goal of being the high-level building block for doing practical, real world data analysis in Python
    \item It's like having a spreadsheet tool built into python
    \item Comes with extensive \underline{\href{https://pandas.pydata.org/docs/}{documentation}}
    \item Has a quick \underline{\href{https://pandas.pydata.org/docs/user_guide/10min.html\#min}{10 minute guide}}
  \end{itemize}
\end{Slide}

%%%%%%%%%%%%%%%%%%%%%%%%%%%%%%%%%%%%%%%%%%%%%%%%%%%%%%%%%%%%%%%%%%%%%%%%%%%%%%%% 
% 
% 
% 
%%%%%%%%%%%%%%%%%%%%%%%%%%%%%%%%%%%%%%%%%%%%%%%%%%%%%%%%%%%%%%%%%%%%%%%%%%%%%%%% 
\begin{Slide}{Data Series}
  \begin{columns}
    \begin{column}{.48\textwidth}
      
      \begin{itemize}
      \item \underline{\href{https://pandas.pydata.org/docs/reference/series.html}{Documentation}}
      \item \underline{\href{https://pandas.pydata.org/docs/user_guide/dsintro.html\#series}{Start Up}}
      \item  a one-dimensional labeled array capable of holding any data type
      \item It's like numpy's ndarray structure
      \end{itemize}
      
      
    \end{column}

    \hfill

    \begin{column}{.48\textwidth}
      Code Example:
      
    \inputminted[firstline=3,
      lastline=23,
      breaklines,
      fontsize=\tiny,
      bgcolor=Background,
      linenos]{python}{../src/data_series.py}
      
      
    \end{column}
  \end{columns}

\end{Slide}

%%%%%%%%%%%%%%%%%%%%%%%%%%%%%%%%%%%%%%%%%%%%%%%%%%%%%%%%%%%%%%%%%%%%%%%%%%%%%%%% 
% 
% 
% 
%%%%%%%%%%%%%%%%%%%%%%%%%%%%%%%%%%%%%%%%%%%%%%%%%%%%%%%%%%%%%%%%%%%%%%%%%%%%%%%% 
\begin{Slide}{DataFrame}
  \begin{columns}
    \begin{column}{.48\textwidth}
      \tiny
      \begin{itemize}
      \item \underline{\href{https://pandas.pydata.org/docs/reference/frame.html}{Documentation}} 
      \item \underline{\href{https://pandas.pydata.org/docs/user_guide/dsintro.html\#dataframe}{Start Up}} 
      \item This is the key data structure of pandas
      \item a 2-dimensional labeled data structure with columns of potentially different types. 
      \item You can think of it like a spreadsheet or SQL table.
      \item Can quickly load a csv file, \underline{\href{https://pandas.pydata.org/docs/reference/api/pandas.read_csv.html}{read\_csv}}  
      \item The \underline{\href{https://pandas.pydata.org/docs/reference/api/pandas.DataFrame.info.html\#pandas.DataFrame.info}{info}}  method prints information about a DataFrame including the index dtype and columns, non-null values and memory usage.
      \item The \underline{\href{https://pandas.pydata.org/docs/reference/api/pandas.DataFrame.columns.html}{columns}} field holds a list of the labels for the columns
      \item The \underline{\href{https://pandas.pydata.org/docs/reference/api/pandas.DataFrame.head.html}{head}} method This function returns the first n rows for the object based on position. It is useful for quickly testing if your object has the right type of data in it.
      \item The \underline{\href{https://pandas.pydata.org/docs/reference/api/pandas.DataFrame.query.html}{query}} method returns the specified columns
      \end{itemize}
      
      
    \end{column}

    \hfill

    \begin{column}{.48\textwidth}
      Code Example:
      
    \inputminted[firstline=3,
      lastline=23,
      breaklines,
      fontsize=\tiny,
      bgcolor=Background,
      linenos]{python}{../src/data_frame.py}
      
      
    \end{column}
  \end{columns}

\end{Slide}

%%%%%%%%%%%%%%%%%%%%%%%%%%%%%%%%%%%%%%%%%%%%%%%%%%%%%%%%%%%%%%%%%%%%%%%%%%%%%%%% 
% 
% 
% 
%%%%%%%%%%%%%%%%%%%%%%%%%%%%%%%%%%%%%%%%%%%%%%%%%%%%%%%%%%%%%%%%%%%%%%%%%%%%%%%% 
\begin{Slide}{Plotting Pandas}
  \begin{columns}
    \begin{column}{.48\textwidth}
      
      \begin{itemize}
      \item Matplotlib has special \underline{\href{https://matplotlib.org/stable/api/dates_api.html}{date}} formatting handlers
      \item There are tools for formatting the date with \underline{\href{https://matplotlib.org/stable/api/figure_api.html\#matplotlib.figure.Figure.autofmt_xdate}{autofmt\_xdate}} 
      \item Pandas can convert to the correct date time formate with \underline{\href{https://pandas.pydata.org/docs/reference/api/pandas.to_datetime.html}{to\_datetime}}
      \item \underline{\href{https://matplotlib.org/stable/api/_as_gen/matplotlib.pyplot.gca.html}{gca}} is Get Current Axis 
      \item \underline{\href{https://matplotlib.org/stable/api/_as_gen/matplotlib.pyplot.gcf.html}{gcf}} is Get Current Figure
      \end{itemize}
      
      
    \end{column}

    \hfill

    \begin{column}{.48\textwidth}
      Code Example:
      
    \inputminted[firstline=9,
      lastline=32,
      breaklines,
      fontsize=\tiny,
      bgcolor=Background,
      linenos]{python}{../src/data_plot.py}
      
      
    \end{column}
  \end{columns}

\end{Slide}

\subsection{Lab}
\begin{Slide}{Lab}
  LAB 3
  \pause
  
  \begin{tcolorbox}[colback=blue!25!white,colframe=blue!75!black,title=THANK YOU]
    THANK YOU FOR ATTENDING! \\
    BEST OF LUCK IN YOUR DATA SCIENCE ENDEAVORS!
  \end{tcolorbox}

\end{Slide}
