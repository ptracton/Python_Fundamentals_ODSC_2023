%%%%%%%%%%%%%%%%%%%%%%%%%%%%%%%%%%%%%%%%%%%%%%%%%%%%%%%%%%%%%%%%%%%%%%%%%%%%%%%% 
% 
% 
% 
%%%%%%%%%%%%%%%%%%%%%%%%%%%%%%%%%%%%%%%%%%%%%%%%%%%%%%%%%%%%%%%%%%%%%%%%%%%%%%%% 
\begin{Slide}{White Space}

  \begin{columns}
    \begin{column}{.48\textwidth}
      
      \begin{itemize}
      \item PEP 8 rules on \underline{\href{https://www.python.org/dev/peps/pep-0008/\#whitespace-in-expressions-and-statements}{White Space}}
      \item Most controversial aspect of Python.
      \item White space used to distinguish blocks
      \item Tab == Space!
      \item Comments and empty lines are ignored
      \item Avoid using Tab
      \item Use and editor/IDE that is Python aware
      \end{itemize}
      
      
    \end{column}

    \hfill

    \begin{column}{.48\textwidth}
      Code Example:
      

      \inputminted[firstline=3,
      lastline=18,
      breaklines,
      fontsize=\tiny,
      bgcolor=Background,
      linenos]{python}{../src/whitespace.py}
      
      
    \end{column}
  \end{columns}


\end{Slide}

%%%%%%%%%%%%%%%%%%%%%%%%%%%%%%%%%%%%%%%%%%%%%%%%%%%%%%%%%%%%%%%%%%%%%%%%%%%%%%%% 
% 
% 
% 
%%%%%%%%%%%%%%%%%%%%%%%%%%%%%%%%%%%%%%%%%%%%%%%%%%%%%%%%%%%%%%%%%%%%%%%%%%%%%%%% 
\begin{Slide}{Coding Standards}
  \begin{itemize}
  \item Python has specified the coding standards for code written in this language
  \item This is \underline{\href{https://www.python.org/dev/peps/pep-0008/}{PEP 8}}
  \item If you are in doubt about a style issue with the way you write Python, refer to PEP 8 and follow it
  \end{itemize}
\end{Slide}

%%%%%%%%%%%%%%%%%%%%%%%%%%%%%%%%%%%%%%%%%%%%%%%%%%%%%%%%%%%%%%%%%%%%%%%%%%%%%%%% 
% 
% 
% 
%%%%%%%%%%%%%%%%%%%%%%%%%%%%%%%%%%%%%%%%%%%%%%%%%%%%%%%%%%%%%%%%%%%%%%%%%%%%%%%% 
\begin{Slide}{Comments}
  \begin{itemize}
  \item Python comments start with '\#'
  \item Multi-line comments can be put in triple quotes  """ THIS IS A COMMENT """ 
  \end{itemize}
\end{Slide}

%%%%%%%%%%%%%%%%%%%%%%%%%%%%%%%%%%%%%%%%%%%%%%%%%%%%%%%%%%%%%%%%%%%%%%%%%%%%%%%% 
% 
% 
% 
%%%%%%%%%%%%%%%%%%%%%%%%%%%%%%%%%%%%%%%%%%%%%%%%%%%%%%%%%%%%%%%%%%%%%%%%%%%%%%%% 
\begin{Slide}{Duck Typing}
  \begin{itemize}
  \item “If it walks like a duck and quacks like a duck...”
  \item No type checking unless absolutely necessary
  \item Python will let you change the data type of a variable at any time!
    \pause
    \begin{tcolorbox}[colback=red!5!white,colframe=red!75!black,title=WARNING]
      BE CAREFUL! \\
      Variables can change type if you are not careful
    \end{tcolorbox}
    
  \end{itemize}
\end{Slide}



%%%%%%%%%%%%%%%%%%%%%%%%%%%%%%%%%%%%%%%%%%%%%%%%%%%%%%%%%%%%%%%%%%%%%%%%%%%%%%%% 
% 
% 
% 
%%%%%%%%%%%%%%%%%%%%%%%%%%%%%%%%%%%%%%%%%%%%%%%%%%%%%%%%%%%%%%%%%%%%%%%%%%%%%%%% 
\begin{Slide}{Variables}
  \begin{itemize}
  \item PEP 8 Rules on \underline{\href{https://www.python.org/dev/peps/pep-0008/\#global-variable-names}{variable names}}
  \item Create variable with assignment:
    \begin{itemize}
    \item  a = 5
    \end{itemize}
    
  \item Variables are references!
  \item Variables are memory managed
  \item Data pointed to by variable is destroyed when variable goes out of scope
  \item Assigning a variable to another creates new reference
    \begin{itemize}
    \item b = a
    \end{itemize}
  \end{itemize}

\end{Slide}


%%%%%%%%%%%%%%%%%%%%%%%%%%%%%%%%%%%%%%%%%%%%%%%%%%%%%%%%%%%%%%%%%%%%%%%%%%%%%%%% 
% 
% 
% 
%%%%%%%%%%%%%%%%%%%%%%%%%%%%%%%%%%%%%%%%%%%%%%%%%%%%%%%%%%%%%%%%%%%%%%%%%%%%%%%% 
\begin{Slide}{Integer Data Types}
  \underline{\href{https://docs.python.org/3/library/stdtypes.html\#numeric-types-int-float-complex}{Official documentation}}\\
  \begin{itemize}
    \item Integers have unlimited precision.
    \item Many built in functions
    \item Tools to convert formats
    \end{itemize}
\end{Slide}


%%%%%%%%%%%%%%%%%%%%%%%%%%%%%%%%%%%%%%%%%%%%%%%%%%%%%%%%%%%%%%%%%%%%%%%%%%%%%%%% 
% 
% 
% 
%%%%%%%%%%%%%%%%%%%%%%%%%%%%%%%%%%%%%%%%%%%%%%%%%%%%%%%%%%%%%%%%%%%%%%%%%%%%%%%% 
\begin{Slide}{Integer Operations}

  \begin{table}[h]
    \begin{tabular}{|l|p{8cm}|} \hline
      x+y    & Addition of x and y \\ \hline
      x-y    & Subtraction of y from x \\ \hline
      x * y  & Multiplication of x and y \\ \hline 
      x/y    & Divides x by y and returns a float \\ \hline
      x//y   & Divides x by y and truncates fractional part to return an integer  \\ \hline
      x\%y   & Returns the modulus (remainder) of x divided by y \\ \hline
      x**y   & Raises x to the power of y \\ \hline
      -x     & negates x  \\ \hline
    \end{tabular}
  \end{table}

\end{Slide}

%%%%%%%%%%%%%%%%%%%%%%%%%%%%%%%%%%%%%%%%%%%%%%%%%%%%%%%%%%%%%%%%%%%%%%%%%%%%%%%% 
% 
% 
% 
%%%%%%%%%%%%%%%%%%%%%%%%%%%%%%%%%%%%%%%%%%%%%%%%%%%%%%%%%%%%%%%%%%%%%%%%%%%%%%%% 
\begin{Slide}{Integer Functions}

  \begin{table}[h]
    \begin{tabular}{|l|p{8cm}|} \hline
      abs(x)    & Returns the absolute value of x \\ \hline
      divmod(x,y)    & Returns the quotient and remainder of x divided by y as a tuple \\ \hline
      pow(x, y)  & Raises x to the power of y, same as x ** y\\ \hline 
      pow(x,y,z)    & Faster alternative to (x ** y)\%z \\ \hline
      round(x, n)   & Returns x rounded to n integeral digits  \\ \hline
      bin(x)  & Returns a string that is a binary representation of x, bin(5) ="0b101" \\ \hline
      hex(x)   &Returns a string that is a hexadecimal representation of x, hex(30) ="0x1e" \\ \hline
      int(x)     & Converts object x to an integer or raises an error\\ \hline
      int(s, base)     & converts string s to an integer  \\ \hline
      oct(x)     & Returns a string that is a octal representation of x, oct(30) =''036" \\ \hline
    \end{tabular}
  \end{table}

\end{Slide}


%%%%%%%%%%%%%%%%%%%%%%%%%%%%%%%%%%%%%%%%%%%%%%%%%%%%%%%%%%%%%%%%%%%%%%%%%%%%%%%% 
% 
% 
% 
%%%%%%%%%%%%%%%%%%%%%%%%%%%%%%%%%%%%%%%%%%%%%%%%%%%%%%%%%%%%%%%%%%%%%%%%%%%%%%%% 
\begin{Slide}{Floating Point}
  \begin{itemize}
  \item Holds double precision values
  \item Range depends on compiler and platform that we are running
  \item Float performs all the same operations as the integer type
  \item float\_var = 1.0
  \end{itemize}
\end{Slide}


%%%%%%%%%%%%%%%%%%%%%%%%%%%%%%%%%%%%%%%%%%%%%%%%%%%%%%%%%%%%%%%%%%%%%%%%%%%%%%%% 
% 
% 
% 
%%%%%%%%%%%%%%%%%%%%%%%%%%%%%%%%%%%%%%%%%%%%%%%%%%%%%%%%%%%%%%%%%%%%%%%%%%%%%%%% 
\begin{Slide}{Strings}
  This is \underline{\href{https://docs.python.org/3/library/stdtypes.html\#text-sequence-type-str}{Python 3 Strings Documentation}}\\
  \begin{itemize}
  \item immutable sequence of Unicode characters
  \item Arbitrary length
  \item Support usual comparison operations 
  \item Can use single, double or tripple quotation marks to create a string
  \item string\_var = 'This is a string'
  \item string\_var = ``This is a string''
  \item string\_var = '''This is a string'''

  \end{itemize}
\end{Slide}

%%%%%%%%%%%%%%%%%%%%%%%%%%%%%%%%%%%%%%%%%%%%%%%%%%%%%%%%%%%%%%%%%%%%%%%%%%%%%%%% 
% 
% 
% 
%%%%%%%%%%%%%%%%%%%%%%%%%%%%%%%%%%%%%%%%%%%%%%%%%%%%%%%%%%%%%%%%%%%%%%%%%%%%%%%% 
\begin{Slide}{String Operations}


  \begin{table}[h]
    \begin{tabular}{|l|p{8cm}|} \hline
      s[x]    & Access index x in the string \\ \hline
      s[:x]    & Access from the start of the string to index x \\ \hline
      s[x:]  & Access from index x to the end of the string \\ \hline 
      s[x:y]    & Access from index x to index y \\ \hline
      s[x:y:z]   & Access from index x to index y byte steps of z\\ \hline
      s1+s2   & Concatenation of 2 strings \\ \hline
      s *n  & Multiply the string n times, creates a new string with n copies of the old string \\ \hline
      len(s) & Length of string s \\ \hline
      c in s & Returns a Boolean True if c is in s \\ \hline
      c not in s & Returns a Boolean True if c is not in s \\ \hline
    \end{tabular}
  \end{table}

\end{Slide}

%%%%%%%%%%%%%%%%%%%%%%%%%%%%%%%%%%%%%%%%%%%%%%%%%%%%%%%%%%%%%%%%%%%%%%%%%%%%%%%% 
% 
% 
% 
%%%%%%%%%%%%%%%%%%%%%%%%%%%%%%%%%%%%%%%%%%%%%%%%%%%%%%%%%%%%%%%%%%%%%%%%%%%%%%%% 
\begin{Slide}{String Methods}

  This is a list of the commonly used methods.\\
  The full list \underline{\href{https://docs.python.org/3/library/stdtypes.html\#str.capitalize}{starts here.}}

  \begin{table}[h]
    \begin{tabular}{|l|p{5cm}|} \hline
      str.capitalize()    & Capitalize the first character \\ \hline
      str.format(*args, **kwargs)    & Reformat the string with the specified parameters \\ \hline
      str.join(iterable)  & Concatenate a list of strings \\ \hline 
      str.lower()    & Returns strings as alll lowercase characters \\ \hline
      str.split(sep=None, maxsplit=-1)   & split the string into a list of strings based on sep delimiter  \\ \hline
      str.strip([chars])   & Return the string with those characters removed. \\ \hline
    \end{tabular}
  \end{table}

\end{Slide}


%%%%%%%%%%%%%%%%%%%%%%%%%%%%%%%%%%%%%%%%%%%%%%%%%%%%%%%%%%%%%%%%%%%%%%%%%%%%%%%% 
% 
% 
% 
%%%%%%%%%%%%%%%%%%%%%%%%%%%%%%%%%%%%%%%%%%%%%%%%%%%%%%%%%%%%%%%%%%%%%%%%%%%%%%%% 
\begin{Slide}{Lists}
  \begin{itemize}
  \item \underline{\href{https://docs.python.org/3/library/stdtypes.html\#lists}{Official Documentation}}
  \item List contains arbitrary objects
  \item Mutable 
  \item Can be sorted
  \item list\_var = []  \# Creates an empty list
  \item list\_var = [1,2,''String'']  \# Creates an list with data
  \end{itemize}
\end{Slide}

%%%%%%%%%%%%%%%%%%%%%%%%%%%%%%%%%%%%%%%%%%%%%%%%%%%%%%%%%%%%%%%%%%%%%%%%%%%%%%%% 
% 
% 
% 
%%%%%%%%%%%%%%%%%%%%%%%%%%%%%%%%%%%%%%%%%%%%%%%%%%%%%%%%%%%%%%%%%%%%%%%%%%%%%%%% 
\begin{Slide}{List Operations}


  \begin{table}[h]
    \begin{tabular}{|l|p{8cm}|} \hline
      s[x]    & Access index x in the list \\ \hline
      s[:x]    & Access from the start of the list to index x \\ \hline
      s[x:]  & Access from index x to the end of the list \\ \hline 
      s[x:y]    & Access from index x to index y \\ \hline
      s[x:y:z]   & Access from index x to index y byte steps of z\\ \hline
      s1+s2   & Concatenation of 2 lists \\ \hline
      s *n  & Multiply the list n times, creates a new list with n copies of the old list \\ \hline
      len(s) & Length of list s \\ \hline
      c in s & Returns a Boolean True if c is in s \\ \hline
      c not in s & Returns a Boolean True if c is not in s \\ \hline
    \end{tabular}
  \end{table}

\end{Slide}

%%%%%%%%%%%%%%%%%%%%%%%%%%%%%%%%%%%%%%%%%%%%%%%%%%%%%%%%%%%%%%%%%%%%%%%%%%%%%%%% 
% 
% 
% 
%%%%%%%%%%%%%%%%%%%%%%%%%%%%%%%%%%%%%%%%%%%%%%%%%%%%%%%%%%%%%%%%%%%%%%%%%%%%%%%% 
\begin{Slide}{List Methods}

  This is a list of the commonly used methods.\\
  The full list \underline{\href{https://docs.python.org/3/tutorial/datastructures.html\#data-structures}{starts here.}}

  \begin{table}[h]
    \begin{tabular}{|l|p{8cm}|} \hline
      list.append(x)    & add x onto the end of the list \\ \hline
      list.extend(L)    & extend the list with a new list L \\ \hline
      list.insert(i,x)    & insert x at position i in the list \\ \hline
      list.pop(i)    & remove the item from position i, if i is not specified, remove the last item \\ \hline
      list.sort()    & sort items in place \\ \hline
    \end{tabular}
  \end{table}

\end{Slide}


%%%%%%%%%%%%%%%%%%%%%%%%%%%%%%%%%%%%%%%%%%%%%%%%%%%%%%%%%%%%%%%%%%%%%%%%%%%%%%%% 
% 
% 
% 
%%%%%%%%%%%%%%%%%%%%%%%%%%%%%%%%%%%%%%%%%%%%%%%%%%%%%%%%%%%%%%%%%%%%%%%%%%%%%%%% 
\begin{Slide}{Dictionaries}
  \begin{itemize}
  \item \underline{\href{https://docs.python.org/3/library/stdtypes.html\#mapping-types-dict}{Official Documentation}}
  \item Dictionaries contain arbitrary key/value pairs
  \item Like an associative array
  \item Keys must be unique and \underline{\href{https://docs.python.org/3/glossary.html\#term-hashable}{hashable}}
  \item Lookup is O(1)
  \item No sorting order
  \item Dictionary is mutable!
  \item Literals:
    \begin{itemize}
    \item dict\_var = \{key1:value1, key2:value2...\}
    \item empty\_dict = \{ \}
    \end{itemize}

  \end{itemize}
\end{Slide}

%%%%%%%%%%%%%%%%%%%%%%%%%%%%%%%%%%%%%%%%%%%%%%%%%%%%%%%%%%%%%%%%%%%%%%%%%%%%%%%% 
% 
% 
% 
%%%%%%%%%%%%%%%%%%%%%%%%%%%%%%%%%%%%%%%%%%%%%%%%%%%%%%%%%%%%%%%%%%%%%%%%%%%%%%%% 
\begin{Slide}{Dictionary Operations and Methods}


  \begin{table}[h]
    \begin{tabular}{|l|p{8cm}|} \hline
      len(d)    & Returns the number of key/value pairs \\ \hline
      value = d[key]    & Returns the value for this key, an error if key does not exist \\ \hline
      d[key] = value & Creates or replaces this key and associates this data with it\\ \hline
      del d[key] & Deletes this key/value \\ \hline
      key in d & Returns True if it is, False if it is not \\ \hline
      d.keys() & Returns a list of keys \\ \hline
      d.values & Returns a list of values \\ \hline
      d.update(dict) & Adds the passed in dictionary to d \\ \hline
      d.items() & Returns a list of key/value tuples \\ \hline
    \end{tabular}
  \end{table}

\end{Slide}


