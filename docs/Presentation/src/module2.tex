\section{Python Fundamentals Module 2}

\subsection{Lecture Summary}
\begin{Slide}{Lecture Summary}
  \begin{itemize}
  \item If-Else
  \item For Loops
  \item While Loops
  \item Functions
  \end{itemize}
\end{Slide}

%%%%%%%%%%%%%%%%%%%%%%%%%%%%%%%%%%%%%%%%%%%%%%%%%%%%%%%%%%%%%%%%%%%%%%%%%%%%%%%% 
% 
% 
% 
%%%%%%%%%%%%%%%%%%%%%%%%%%%%%%%%%%%%%%%%%%%%%%%%%%%%%%%%%%%%%%%%%%%%%%%%%%%%%%%% 
\begin{Slide}{If Statements}
  \begin{columns}
    \begin{column}{.48\textwidth}
      
      \begin{itemize}
      \item \underline{\href{https://docs.python.org/3/tutorial/controlflow.html\#if-statements}{Official Documentation}}
      \item elif and else are optional
      \end{itemize}
      
      
    \end{column}

    \hfill

    \begin{column}{.48\textwidth}
      Code Example:
      
    \inputminted[firstline=3,
      lastline=11,
      breaklines,
      fontsize=\tiny,
      bgcolor=Background,
      linenos]{python}{../src/ifexample.py}
      
      
    \end{column}
  \end{columns}

\end{Slide}

%%%%%%%%%%%%%%%%%%%%%%%%%%%%%%%%%%%%%%%%%%%%%%%%%%%%%%%%%%%%%%%%%%%%%%%%%%%%%%%% 
% 
% 
% 
%%%%%%%%%%%%%%%%%%%%%%%%%%%%%%%%%%%%%%%%%%%%%%%%%%%%%%%%%%%%%%%%%%%%%%%%%%%%%%%% 
\begin{Slide}{For Statements}
  \begin{columns}
    \begin{column}{.48\textwidth}
      
      \begin{itemize}
      \item \underline{\href{https://docs.python.org/3/tutorial/controlflow.html\#for-statements}{Official Documentation}}
      \item Do not specify start, stop and step like C
      \item Iterate over a list, file or string
      \item \underline{\href{https://docs.python.org/3/library/stdtypes.html\#range}{range}} function returns a list of numbers to iterate over
      \item break will exit the loop 
      \item else: will execute at the end of the loop if the exit is normal (not break)
      \end{itemize}
      
      
    \end{column}

    \hfill

    \begin{column}{.48\textwidth}
      Code Example:
      
    \inputminted[firstline=3,
      lastline=20,
      breaklines,
      fontsize=\tiny,
      bgcolor=Background,
      linenos]{python}{../src/forexample.py}
      
      
    \end{column}
  \end{columns}
\end{Slide}

%%%%%%%%%%%%%%%%%%%%%%%%%%%%%%%%%%%%%%%%%%%%%%%%%%%%%%%%%%%%%%%%%%%%%%%%%%%%%%%% 
% 
% 
% 
%%%%%%%%%%%%%%%%%%%%%%%%%%%%%%%%%%%%%%%%%%%%%%%%%%%%%%%%%%%%%%%%%%%%%%%%%%%%%%%% 
\begin{Slide}{While Statements}
  \begin{columns}
    \begin{column}{.48\textwidth}
      
      \begin{itemize}
      \item \underline{\href{https://docs.python.org/3/reference/compound_stmts.html\#the-while-statement}{Official Documentation}}
      \item As long as the expression in the while statement evaluates to True, the loop will process
      \item break can exit the loop 
      \item else: will execute at the end of the loop if the exit is normal (not break)
      \end{itemize}
      
      
    \end{column}

    \hfill

    \begin{column}{.48\textwidth}
      Code Example:
      

    \inputminted[firstline=3,
      lastline=22,
      breaklines,
      fontsize=\tiny,
      bgcolor=Background,
      linenos]{python}{../src/whileexample.py}

      
      
    \end{column}
  \end{columns}
\end{Slide}


%%%%%%%%%%%%%%%%%%%%%%%%%%%%%%%%%%%%%%%%%%%%%%%%%%%%%%%%%%%%%%%%%%%%%%%%%%%%%%%% 
% 
% 
% 
%%%%%%%%%%%%%%%%%%%%%%%%%%%%%%%%%%%%%%%%%%%%%%%%%%%%%%%%%%%%%%%%%%%%%%%%%%%%%%%% 
\begin{Slide}{Basic Exceptions}
  \begin{columns}
    \begin{column}{.48\textwidth}
      
      \begin{itemize}
      \item Exceptions are pythons way of handling errors
      \item \underline{\href{https://docs.python.org/3/library/exceptions.html}{Exception Documentation}}
      \item \underline{\href{https://docs.python.org/3/tutorial/errors.html}{Exception Tutorial}}
      \item You must have try and except, all other elements are optional
      \item Each exception block is tried in order
      \item Use liberally and do not let errors go silently
      \item A blank except: will catch any exception.
      \end{itemize}
      
      
    \end{column}

    \hfill

    \begin{column}{.48\textwidth}
      \pause

      \inputminted[firstline=3,
      lastline=27,
      breaklines,
      fontsize=\tiny,
      bgcolor=Background,
      linenos]{python}{../src/exceptions.py}
            
      
    \end{column}
  \end{columns}
\end{Slide}



%%%%%%%%%%%%%%%%%%%%%%%%%%%%%%%%%%%%%%%%%%%%%%%%%%%%%%%%%%%%%%%%%%%%%%%%%%%%%%%% 
% 
% 
% 
%%%%%%%%%%%%%%%%%%%%%%%%%%%%%%%%%%%%%%%%%%%%%%%%%%%%%%%%%%%%%%%%%%%%%%%%%%%%%%%% 
\begin{Slide}{Functions}
  \begin{columns}
    \begin{column}{.48\textwidth}
      \tiny
      \begin{itemize}
      \item Functions are a way of breaking up the code into more manageable pieces.
      \item There are 4 types of Functions
        \begin{enumerate}
          \tiny
          \pause
        \item Global are available to everyone in the module
          \pause
        \item Local are functions inside of functions
          \pause
        \item Lambda are limited functions that are created just in time to use them
          \pause
        \item Methods are associated with a specific data type and part of Object Oriented Programming
        \end{enumerate}
      \end{itemize}
      
      
    \end{column}

    \hfill

    \begin{column}{.48\textwidth}
      \pause 
      Psuedo Code Example:
            
      \inputminted[firstline=3,
      lastline=10,
      breaklines,
      fontsize=\tiny,
      bgcolor=Background,
      linenos]{python}{../src/psuedo.py}
      
      
    \end{column}
  \end{columns}
\end{Slide}

%%%%%%%%%%%%%%%%%%%%%%%%%%%%%%%%%%%%%%%%%%%%%%%%%%%%%%%%%%%%%%%%%%%%%%%%%%%%%%%% 
% 
% 
% 
%%%%%%%%%%%%%%%%%%%%%%%%%%%%%%%%%%%%%%%%%%%%%%%%%%%%%%%%%%%%%%%%%%%%%%%%%%%%%%%% 
\begin{Slide}{Names and DocStrings}

  \begin{itemize}
  \item PEP 8 rules on \underline{\href{https://www.python.org/dev/peps/pep-0008/\#function-names}{Function Names}}
  \item Use good clear names for the functions, should indicate what the function does
  \item Avoid abbreviations
  \item Docstrings are comments that come right after the def line of the function.  It should indicate what the function does and how to use it.  
\item Docstrings have their own PEP, \underline{\href{https://www.python.org/dev/peps/pep-0257/}{PEP 257}}
\item Although \underline{\href{https://www.python.org/dev/peps/pep-0008/\#documentation-strings}{PEP 8}} also has some thoughts
  \end{itemize}
\pause
\begin{alertblock}
{All functions in your labs must have doc strings!}
\end{alertblock}


\end{Slide}

%%%%%%%%%%%%%%%%%%%%%%%%%%%%%%%%%%%%%%%%%%%%%%%%%%%%%%%%%%%%%%%%%%%%%%%%%%%%%%%% 
% 
% 
% 
%%%%%%%%%%%%%%%%%%%%%%%%%%%%%%%%%%%%%%%%%%%%%%%%%%%%%%%%%%%%%%%%%%%%%%%%%%%%%%%% 
\begin{Slide}{Function Arguments}
  \begin{itemize}
  \item Does not type check
  \begin{itemize}
  \item you can manually check if you wish
  \end{itemize}
\item All values are passed by reference
 \begin{itemize}
  \item immutable types can't be modified by a function
  \end{itemize}
\item Arguments are local variables 
\item Arguments go out of scope when functions return
\item Without a return statement, the function returrns \textbf{None}
  \end{itemize}
\end{Slide}

%%%%%%%%%%%%%%%%%%%%%%%%%%%%%%%%%%%%%%%%%%%%%%%%%%%%%%%%%%%%%%%%%%%%%%%%%%%%%%%% 
% 
% 
% 
%%%%%%%%%%%%%%%%%%%%%%%%%%%%%%%%%%%%%%%%%%%%%%%%%%%%%%%%%%%%%%%%%%%%%%%%%%%%%%%% 
\begin{Slide}{Default Arguments}
 \begin{columns}
    \begin{column}{.48\textwidth}
      
      \begin{itemize}
      \item Specify a default value for a function argument 
\item If no value is given for this parameter when the function is called, the default value is used
\item This is very handy!
     
      \end{itemize}
      
      
    \end{column}

    \hfill

    \begin{column}{.48\textwidth}
\pause
      Code Example:
      
      \inputminted[firstline=3,
      lastline=23,
      breaklines,
      fontsize=\tiny,
      bgcolor=Background,
      linenos]{python}{../src/functions.py}

      
      
    \end{column}
  \end{columns}

\end{Slide}

%%%%%%%%%%%%%%%%%%%%%%%%%%%%%%%%%%%%%%%%%%%%%%%%%%%%%%%%%%%%%%%%%%%%%%%%%%%%%%%% 
% 
% 
% 
%%%%%%%%%%%%%%%%%%%%%%%%%%%%%%%%%%%%%%%%%%%%%%%%%%%%%%%%%%%%%%%%%%%%%%%%%%%%%%%% 
\begin{Slide}{Named Arguments}
  \begin{columns}
    \begin{column}{.48\textwidth}
      
      \begin{itemize}
      \item Use name arguments to comment the code (use good names!)
      \item Can be used to skip arguments
      \end{itemize}
      
      
    \end{column}

    \hfill

    \begin{column}{.48\textwidth}
\pause
      Code Example:
            
   \inputminted[firstline=3,
      lastline=27,
      breaklines,
      fontsize=\tiny,
      bgcolor=Background,
      linenos]{python}{../src/functions.py}
      
      
    \end{column}
  \end{columns}


\end{Slide}


%%%%%%%%%%%%%%%%%%%%%%%%%%%%%%%%%%%%%%%%%%%%%%%%%%%%%%%%%%%%%%%%%%%%%%%%%%%%%%%% 
% 
% 
% 
%%%%%%%%%%%%%%%%%%%%%%%%%%%%%%%%%%%%%%%%%%%%%%%%%%%%%%%%%%%%%%%%%%%%%%%%%%%%%%%% 
\begin{Slide}{Files}

  \begin{itemize}
  \item \underline{\href{https://docs.python.org/3.4/tutorial/inputoutput.html\#reading-and-writing-files}{Official Documentation}}
  \item open(path [,mode]) – returns file object
  \item Mode specifies file mode
    \begin{itemize}
    \item r  open for read
    \item w  open for write, creates file if needed
    \item a  open for append
    \item r+ open for update (r+w)
    \item w+ open for update (r+w), truncate!
    \item rb, wb, ab, rb+, wb+  open in binary mode
    \end{itemize}
  \item Mode defaults to 'r'
  \end{itemize}

\end{Slide}

%%%%%%%%%%%%%%%%%%%%%%%%%%%%%%%%%%%%%%%%%%%%%%%%%%%%%%%%%%%%%%%%%%%%%%%%%%%%%%%% 
% 
% 
% 
%%%%%%%%%%%%%%%%%%%%%%%%%%%%%%%%%%%%%%%%%%%%%%%%%%%%%%%%%%%%%%%%%%%%%%%%%%%%%%%% 
\begin{Slide}{File Reading}
  \begin{columns}
    \begin{column}{.48\textwidth}
      \begin{itemize}
      \item read([size]) – read size bytes
        \begin{itemize}
        \item Returns string
        \item <size> defaults to infinite
        \end{itemize}
      \item readline([size]) – read line
      \item readlines() - read all lines

      \end{itemize}
    \end{column}

    \hfill

    \begin{column}{.48\textwidth}

      \pause

      \inputminted[firstline=3,
      lastline=27,
      breaklines,
      fontsize=\tiny,
      bgcolor=Background,
      linenos]{python}{../src/filer.py}
      
    \end{column}
  \end{columns} 
\end{Slide}

%%%%%%%%%%%%%%%%%%%%%%%%%%%%%%%%%%%%%%%%%%%%%%%%%%%%%%%%%%%%%%%%%%%%%%%%%%%%%%%% 
% 
% 
% 
%%%%%%%%%%%%%%%%%%%%%%%%%%%%%%%%%%%%%%%%%%%%%%%%%%%%%%%%%%%%%%%%%%%%%%%%%%%%%%%% 
\begin{Slide}{File Writing}
  \begin{columns}
    \begin{column}{.48\textwidth}
      \begin{itemize}
      \item write(str) – writes string to file
      \item writelines(list) – writes sequence of strings to
      \item file flush () - flushes write buffer
      \end{itemize}
    \end{column}

    \hfill

    \begin{column}{.48\textwidth}

      \pause

      \inputminted[firstline=3,
      lastline=27,
      breaklines,
      fontsize=\tiny,
      bgcolor=Background,
      linenos]{python}{../src/filew.py}
      
    \end{column}
  \end{columns} 
\end{Slide}

%%%%%%%%%%%%%%%%%%%%%%%%%%%%%%%%%%%%%%%%%%%%%%%%%%%%%%%%%%%%%%%%%%%%%%%%%%%%%%%% 
% 
% 
% 
%%%%%%%%%%%%%%%%%%%%%%%%%%%%%%%%%%%%%%%%%%%%%%%%%%%%%%%%%%%%%%%%%%%%%%%%%%%%%%%% 
\begin{Slide}{File Other}

  \begin{itemize}
  \item close() - flush buffer and close file
  \item fileno() - get OS file handle
  \item tell() - get file position
  \item seek(offset [,whence]) – set file position
    \begin{itemize}
    \item os.SEEK\_SET
    \item os.SEEK\_CUR
    \item os.SEEK\_END
    \end{itemize}
  \item truncate([size]) – cut off file length
  \end{itemize}

\end{Slide}


%%%%%%%%%%%%%%%%%%%%%%%%%%%%%%%%%%%%%%%%%%%%%%%%%%%%%%%%%%%%%%%%%%%%%%%%%%%%%%%% 
% 
% 
% 
%%%%%%%%%%%%%%%%%%%%%%%%%%%%%%%%%%%%%%%%%%%%%%%%%%%%%%%%%%%%%%%%%%%%%%%%%%%%%%%% 
\begin{Slide}{Printing to a File}
  \begin{columns}
    \begin{column}{.48\textwidth}
      \begin{itemize}
      \item \underline{\href{https://docs.python.org/release/3.1/library/functions.html\#print}{Official Documentation}}
      \item file defaults to stdout
      \item it can be changed to any file you want
      \end{itemize}
    \end{column}

    \hfill

    \begin{column}{.48\textwidth}

      \pause

      \inputminted[firstline=3,
      lastline=27,
      breaklines,
      fontsize=\tiny,
      bgcolor=Background,
      linenos]{python}{../src/filep.py}

      
    \end{column}
  \end{columns} 
\end{Slide}


\subsection{Lab}
\begin{Slide}{Lab}
  LAB 2
\end{Slide}
